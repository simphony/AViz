\documentclass[11pt]{article}
\usepackage{palatino}
\usepackage{html}

\bodytext{BGCOLOR="#FFFFFF"}	% this is ignored by latex 

\setlength{\parskip}{0.2cm}
\begin{document}

\title{AViz Atomic Visualization User's Guide}

\author{Anastasia Sorkin and Geri Wagner}
\date{\today}
\maketitle

\tableofcontents

\section{What Can AViz Be Used For?}

AViz is a program to visualize data sets generated in atomistic 
simulations.  In such simulations, one usually operates with 
a number of particles, each of which representing a particle.  The 
result of the simulation typically consists of data files containing
the coordinates of all the simulated particles.  

You can use AViz to render the simulated particles as little spheres, cubes, or 
dots.  It can also be used to render atomistic spins, or any other kind of 
vector.    Several different shapes and qualities can be used, and 
colors can be assigned to the rendered objects in many different ways.

\subsection{License and Distribution}

AViz is distributed under the GNU General Public License.  

This program is free software; you can redistribute it and/or
modify it under the terms of the GNU General Public License
as published by the Free Software Foundation; either version 2
of the License, or (at your option) any later version.

This program is distributed in the hope that it will be useful,
but WITHOUT ANY WARRANTY; without even the implied warranty of
MER\-CHANTABI\-LITY or FITNESS FOR A PARTICULAR PURPOSE.  See the
GNU General Public License for more details.

You should have received a copy of the GNU General Public License
along with this program; if not, write to the Free Software
Foundation, Inc., 59 Temple Place - Suite 330, Boston, MA  02111-1307, USA.

A full version of the license and distribution regulations is printed 
if you start AViz by typing {\tt aviz -license} and {\tt aviz -distribution}.

\section{Obtaining and Installing AViz}

To download the AViz software, go to the AViz home page at 
\begin{center}
{\tt http://phycomp.technion.ac.il/~aviz} 
\end{center}
and find the section labeled 
''Download''.  You will obtain a compressed archive file.  

Uncompress and unpack the archive.  Move into the {\tt aviz} directory 
and execute the following steps:

\begin{verbatim}
./configure
make
make install
\end{verbatim}

If all went well, you should now have your copy of AViz installed.  

If not, one likely reason is that some of the software required to compile
and run AViz is missing on your system.  

Software that may be missing includes:
\begin{itemize}
\item The Qt development library -- you may obtain it 
at 
\begin{center}
{\tt http://www.trolltech.com} 
\end{center}
(Linux Users may download and install as 
a RPM file)
\item The {\tt tmake} Qt makefile utility -- you may obtain it 
at 
\begin{center}
{\tt http://www.trolltech.com} 
\end{center}
(Linux Users may download and install 
as a RPM file)
\item The OpenGL graphics library -- you may obtain a free clone called 
{\tt Mesa} from numerous sites on the web (Linux Users may download and install as a RPM file)
\end{itemize}


\section{Starting AViz}

AViz is started by typing {\tt aviz}.  If a data file name is added, AViz 
will automatically read the file and render the data.

\subsection{Command Line Arguments}

AViz accepts standard X11 command line arguments.  In addition, the 
following command line arguments given in table \ref{tab1} are defined.

\begin{table}[h]
\begin{tabular}{|ll|}
\hline
{\tt -ar}   &  Start auto rotation \\
{\tt -as}   &  Start auto spinning \\
{\tt -at}   &  Start auto tilting \\
{\tt -az}   &  Start auto zooming \\
{\tt -asm}   & Start in auto snap mode\\
{\tt -fl [listfile]}   		& Start using specified file list \\
{\tt -rm [atom|spin|liquidcrystal|polymer|pore]} & Set render mode\\
{\tt -rs [dots|lines|cubes|cones|cylinders|spheres]} & Set render style \\
{\tt -rq [low|high|final]    } & Set render quality \\
{\tt -si [percentage]              } & Set render size in percentage \\
{\tt -snap 		           } & Take a screen shot of the \\
{                                  } & rendering at startup \\
{\tt -snapq 		           } & Take a screen shot of the \\
{                                  } & rendering at startup and exit\\
{\tt -v(ersion)                    } & Print version number and exit \\
{\tt -vpm [viewParamfile]          } & Use view parameters in specified \\
				     & file at startup \\
{\tt -watch [XYZfile]} 		     & Update rendering upon changes in \\
                                     & specifed coordinate file \\
{\tt -license} 				& Print GNU license applying to this \\
				     & software \\
{\tt -distribution} 		& Print GNU regulations under applying \\
				     & to this software \\
{\tt -usage|-h(elp)} 		& Print this message and exit \\
\hline
\end{tabular}
\label{tab1}
\caption{AViz Command line arguments}
\end{table}

\section{Data File Formats}

AViz can read three different kinds of files: coordinate data files, 
lists of coordinate data files, and view parameter files.

\subsection{Coordinate Data Files}
The format for the coordinate data files (''XYZ files'') is as follows: 

\indent
  1. line: an integer indicating the number of data points \newline
\indent
  2. line: an arbitrary string identifying the data  \newline
\indent
  following lines: all of the form XX float1 float2 float3 ..... 

Each of the lines corresponds to one particle. A one- or two-character string 
is used to label the particle, as in 'O' or 'Si'.  You may choose any 
combination of characters you like.  The following three 
floating point numbers represent the x-, y-, and z-coordinate of the particle. 
Up to eight additional floating point numbers can be provided. These 
additional numbers can be used to visualize properties such as velocity, 
energy, etc. 

AViz recognizes some particular labels such as 'Al' and 'Au'.  For these 
atom types, a default color is defined.  For many other elements, and 
for atom types such as 'X3', no default color is defined.  AViz will 
use a blue color to render particles of ''unknown'' type.

Here is a short example data file of the most simple form, containing 
only the three coordinate columns:
\begin{verbatim}
8
## Cube
C  1.0 0.0 0.0
C  1.0 1.0 0.0
C  0.0 1.0 0.0
C  0.0 1.0 0.0
C  1.0 0.0 1.0
C  1.0 1.0 1.0
C  0.0 1.0 1.0
C  0.0 1.0 1.0
\end{verbatim}

Here is a short example data file with one additional column of data: 
\begin{verbatim}
6
## Aluminum oxide
O  -0.083333 0.250000 0.583333 0.433013
O  0.083333 -0.250000 -0.583333 -0.433013
O  0.250000 0.583333 -0.083333 0.433013
O  -0.250000 -0.583333 0.083333 -0.433013
Al 0.583333 0.583333 0.583333 1.010363
Al -0.583333 -0.583333 -0.583333 -1.010363
\end{verbatim}

If the data points should be rendered as vectors, each vector is defined by 
six coordinates.  The first three give the location of the vector, and the 
second three give the spin magnitude and direction.  A sample data file might 
look like this:

\begin{verbatim}
3
## Spins pointing in z-direction
Sp 0.0 0.0 0.0 0.0 0.0 1.0
Sp 1.0 0.0 0.0 0.0 0.0 0.5
Sp 2.0 0.0 0.0 0.0 0.0 0.25
\end{verbatim}

\subsection{Lists of Coordinate Data Files}
\label{fileList}

A list of data files is exactly what you would expect, and may look 
as follows:

\begin{verbatim}
aluOx1.xyz
aluOx2.xyz
aluOx3.xyz
aluOx4.xyz
\end{verbatim}

When opening such a file, a small control board will pop up that lets you 
cycle through these files, in single steps or continuously; see section
\ref{fileListDialog}.  

\subsection{View Parameter File}

AViz can create binary files that contain all view parameters, such as 
viewing angles, atom colors, annotation, etc.  These files can be saved
explicitely (using the {\bf File/Save} menu entry) or implicitely (using the 
{\bf File/Set Defaults} menu entry). 


\subsection{Particle Parameter File}
\label{particleParameterFile}
AViz creates a binary file called {\em aviz.particle} in a hidden directory 
{\tt .aviz} in your home directory.  This file contains information about 
settings that apply to a given particle type -- which color should be used 
in the rendering, for instance.  

The particle parameter file is updated automatically each time you 
change particle type settings.

\section{AViz Main Window}

AViz starts up with two windows, a main window and a graphics window.

The main window includes a menu bar, switches to change the rendering 
stye, buttons to create snapshots of the current rendering, and a status 
line.

The menu bar entries are discussed in the following sections.  In this section
the rendering controls are discussed.

The {\bf Mode} combo bar is used to define the nature of your particles.  
The current range of possible modes includes atom, spin, liquid crystal, 
polymer, and pore.  The effect of changing modes is mainly that it gives
you different options when it comes to choosing render style and color 
assignment.

Depending on the setting of the combo box labeled {\bf Style}, particles are
rendered either as dots, squares, spheres, cones, cylinders, or pins.  Note
that not all selections are accessible at all times -- it depends on the render
mode setting.  A text field labeled {\bf Size} is used to adjust the size of
each object globally.   

The {\bf Quality} combo box is used to control render quality.  The idea is to
find a good viewing position using low quality, and switch to high or even
final only then. 

At any stage, an image file representing the current rendering can be created,
by pressing the {\bf Snap} button.  The image file is saved in the current
directory.  The image is encoded in the PNG format and
is called {\tt file.NNNN.png}, where {\tt file} is the filename of the data file
containing the atom positions, and NNNN is a four-digit number which 
increments.  The idea is to prevent overwriting of existing image files.

When the adjacent {\bf Auto Snap} button is depressed, AViz will take a
snapshot whenever the rendering was updated. 

\subsection{File Menu}

\subsubsection{Open...}
\label{fileOpen}
Click on {\bf Open} to open either a coordinate data file in XYZ-format, a list
of coordinate data files,  or a view parameter file.   

When opening a coordinate data file, the file browser will sport an extra 
button labeled {\bf Generate File List}.  The button is associated with 
a convenience function that wil create a file list (see section \ref{fileList})
including all coordinate files in the current directories.  Coordinate files 
are recognized if their filename contains the suffix {\tt .xyz}.

\subsubsection{Save ViewParam File...}

Click on {\bf Save ViewParam File...} to save the current view parameters in a
view parameter file.   

\subsubsection{File List...}

Click on {\bf File List...} to pop up a dialog window that lets you leave
through the files in the current file list (see section \ref{fileListDialog}).

\subsubsection{Animation...}

Click on {\bf Animation...} to pop up a dialog window that lets you generate an
AnimatedGIF file based on image files in PNG-format; see section
\ref{creatingAnimatedGIF}.  This is a convenience function of experimental
character.  It relies on the image processing program {\tt convert} which is
quite common on Linux systems.

\subsubsection{Watch XYZ File}

The {\bf Watch XYZ File} button causes watching of the currently rendered
coordinate file: whenever the file changes, the graphics display is updated, 
showing the current data.  File changes are checked every 10 seconds.  

If the {\bf AutoSnap} button is depressed, you will obtain a PNG snapshot
upon every update.  This may be a convenient way to generate a visualization
of the dynamics in your simulation; see also section \ref{watching}.

\subsubsection{Snap PNG File...}

Click on {\bf Snap PNG File...} to launch a file browser and create a 
PNG image file of the current rendering in a specific directory. 

Alternatively, you may simply press then {\bf Snap} button to generate a 
PNG file in the current directory.

\subsubsection{Set Default View Param...}

Click on {\bf Set Default View Param } to save the current view parameters in a
view parameter file called {\tt aviz.vpm} in a hidden directory {\tt .aviz} in
your home directory. If this file is present in the directory in which AViz is
started the next time, AViz will automatically read and set these parameter
values.    

The view parameters include parameters such as view point, lighting, and 
annotation.  They do not include parameters related to the rendering of 
a class of particles, such as the color of the 'Al' atoms.  The idea is that
while you may want to change the view point frequently, you may not
want to change particle colors.

\subsubsection{Exit}

Click on {\bf Exit} to exit.

\subsection{Elements Menu}

\subsubsection{Atoms...}

This menu entry is active if the render mode is set to ''Atom''.  

Click on {\bf Atoms/Molecules...} to open the atom type dialog window to set
some properties that affect the rendering of atoms of a given type, such
as {\tt Al}.   

A selector is used to select the atom type to which the settings apply.

A toggle button labeled {\bf Show Atoms} can be used to switch off the 
rendering of all atoms of the selected atom type.

A color bar shows the color, or spectrum of colors, that is currently used 
to render atoms of the given type.

If rendering is not disabled, you can adjust the size of the atoms of this 
type, relative to the default size.  This is done using three radio buttons
labeled {\bf Regular}, {\bf Smaller} and {\bf Larger}.

The coloring of the atoms of the selected atom type can be controlled using the
radio buttons labeled {\bf Type}, {\bf Position}, {\bf Property}, and 
{\bf ColorCode}. 

If ''Type'' is chosen, all atoms of the selected type are colored the 
same color (which is set using the {\bf Set Color...} button). 

If ''Position'' is chosen, the atoms of the selected type are colored according
to their coordinate values.  Radio buttons labeled {\bf X}, {\bf Y}, and {\bf
Z} are used to select the reference coordinate axis.  (The color scale that is
used is set using the {\bf Set Color...} button.)

If ''Property'' is chosen, the atoms of the selected type are colored according
to their property values given in the input data file in addition to their
coordinate values.  Radio buttons labeled {\bf 1}, {\bf 2}, etc.  are used to
select the reference property.  (The color scale that is used is set using the
{\bf Set Color...} button.)  This feature can be used to visualize scalar 
particle properties such as kinetic energy that are given together with the 
particle position in the coordinate file.

If ''ColorCode'' is chosen, you assume complete control over the coloring 
by specifying property colums that are interpteted as red--green--blue color 
values, ranging from 0.0 to 1.0.  By default, the first three property 
values are taken as red, green, and blue values, respectively.   For instance, 
if the default settings are chosen and the entry of the first particle is
\begin{verbatim}
Al  3.57.0 24.34 0.249		0.0	1.0	0.0
\end{verbatim}
that particle is rendered using a strong green color.

The settings are effective after {\bf Apply} or {\bf Done} is pressed.

Click on the {\bf Bonds...} button in the atom type dialog, or select the {\bf
Bonds...} submenu entry to open the bond dialog window.  This window is used to
control the rendering of bonds between atoms of given atom types, such as {\tt
Al} and {\tt O}.  

Two selectors, {\bf From Particle Type} and {\bf To Particle Type}, are used to 
select the atom types to which the settings apply. 

Radio buttons labeled {\bf Present} and {\bf Absent} are used to enable and
disable the rendering of bonds between the atoms of the specified types.

If bond drawing is enabled, the color of the bond can be set using the radio
buttons in the {\bf Color} box.  

The thickness can be set using the radio buttons in the {\bf Thickness} box.
If {\bf Line} is chosen, the bonds will be rendered as lines; this is the
fastest rendering mode.  Other thicknesses, in combination with high render
quality, will give you bonds that are rendered as cylinders.  

Depending on the view point, the rendering can be improved by activating
anti-aliasing.  Use the checkbox labeled {\bf AntiAliasing} to control 
this feature.

A text window labeled {\bf Emission} is provided to set the emission light
quality of the bond.  In certain circumstances bond light emission can improve
the visual appearance of the rendering.

Most important are the minimum length and maximum length parameters, to be
entered in text windows labeled {\bf Min Length} and {\bf Max Length}.  The
lengths are given in the same units as the particle coordinates.  Only
those bonds will be rendered whose length is within the limits specified.

The setting is effective after {\bf Apply} or {\bf Done} is pressed.

\subsubsection{Spins...}

This menu entry is active if the render mode is set to ''Spin''.

Click on {\bf Spins...} to open a small dialog window to set the properties
that affect the rendering of spins or other types of vectors.  The vectors
are rendered as lines or cylinders with a sphere at the end.

The dialog operates similar to the atom property dialog.  In addition,
you may set some settings that affect the rendering of the spin foots.

The thickness can be set using the radio buttons in the {\bf Thickness} box.
If {\bf Line} is chosen, the spins will be rendered as lines; this is the
fastest rendering mode.  In combination with high rendering quality, the
other options will give you cylinders rather than simple lines.

A text window labeled {\bf Emission} is provided to set the emission light
quality of the spin foot.  In certain circumstances light emission can improve
the visual appearance of the rendering.

The setting is effective after {\bf Apply} or {\bf Done} is pressed.

\subsubsection{Liquid Crystals...}

This menu entry is active if the render mode is set to ''LiquidCrystal''.

Click on {\bf Liquid Crystals...} to open a small dialog window to set the
properties that affect the rendering of liquid crystal molecules of a 
given type.  The liquid crystal molecules are rendered as lines, cones, 
or cylinders, with the startpoint and endpoint given by six columns of 
numbers.  

The dialog operates similar to the spin property dialog.  For liquid crystals,
you can assign up to three colors to a given molecule type.  For instance, 
if you assign red and blue to a type and choose the ''cylinder'' rendering
style, half of the cylinder is rendered in red, and the other half in blue. 
In this way, you can visualize orientation patterns.

The setting is effective after {\bf Apply} or {\bf Done} is pressed.

\subsubsection{Polymers...}

The ''Polymer'' rendering mode is almost identical to the ''Atom'' rendering
mode.  The polymer type dialog that comes up when you choose the 
{\bf Polymers...} submenu entry produces a dialog that is identical to the 
atom type dialog.  The polymer bond dialog, popping up when you select the 
{\bf Bonds...} submenu entry is also identical with the atom bond dialog. 

The difference is that polymer bonds are only drawn between particles that 
belong to the same molecule.  The fourth column in the coordinate file is 
expected to contain an integer value indicating, for each particle, the 
molecule to which it belongs.  For instance, two atoms belonging to the 
first molecule may be entered as follows:
\begin{verbatim}
C  0.0 0.0 0.0 1
H  0.5 0.0 0.0 1
\end{verbatim}

\subsubsection{Pores...}

The ''Pore'' rendering mode operates similar to the liquid crystal 
rendering mode.  You may draw lines or cylinders from starting point to 
end point.  If coloring is according to particle type, up to two colors 
can be assigned to each particle type.   

\subsubsection{Lights...}

Click on the {\bf Lights...} menu entry to open a small dialog window to
control the lighting of the rendering.   Eight spotlights can be activated
independently, shining from different positions such as ''Top'', ''Bottom
Right'' etc.  This is done using toggle buttons labeled accordingly.

The distance from which a given light source shines is set by means of the 
slider labeled {\bf Depth}.  A combo box is used to specify light sources 
for which the setting applies.  The greater the depth, the sharper the 
distinction between bright and dark regions.

A slider labeled {\bf Global Ambient Light} is used to set the level of 
ambient light.  A high level lightens up the shadows in the rendering. 

The shininess of the rendered atoms is set using a second slider labeled 
{\bf Surface Shininess}.  A high level results in a small bright spot on 
each of the atoms.  Note that this effect is most prominent in ''Final'' 
rendering mode.

The setting is effective after {\bf Apply} or {\bf Done} is pressed.

\subsubsection{Annotation...}

Click on the {\bf Annotation...} menu entry to open a small dialog window to
specify an annotation to the rendering.  The annotation string is entered in a
one-line text window.  

A toggle button is used to switch on or off annotation display.

A spin box is used to set the size of the annotation -- currently between 1 and
4.  (Note that the color is yellow by default).  Two other spin boxes are used
to specify the location of the annotation.  The location is given in screen
coordinates (pixels).  (If you specify a location that is off the screen, you
won't see your annotation.)

\subsection{View Menu}

\subsubsection{Set Viewpoint...}

The {\bf Set Viewpoint...} menu entry opens a submenu pane.  Select 
{\bf Explicit...} to open a dialog that lets you specify the view point
in great detail.

In the dialog are given four text fields, labeled ''{\bf Phi}'', ''{\bf
Theta}'', ''{\bf Chi}'', and ''{\bf Eye}'' The first three fields show anglular
values (in deg) that are used to rotate the set of atoms to be rendered.  The
rotation is based on Euler angles phi, theta, chi: First the set is rotated
about its z-axis by phi degrees.  Then the set is rotated about its new y-axis
by theta degrees.  Then the set is rotated about its new z-axis by chi degrees.

The text field labeled ''{\bf Eye}'' is used to adjust the distance from the
view point to the set.

The setting is effective after {\bf Apply} or {\bf Done} is pressed.

Choose {\bf Default View} to assume the view parameters defined as default.
Alternatively, choose {\bf View XY+} to view along the positive $z$-axis, {View
XY-} to view along the negative $z$-axis, and so on.

\subsubsection{Clipping...}

Click on {\bf Clipping...} to open a small dialog window that allows the manual
setting of the near clip plane and the far clip plane.  Clip planes restrict
the volume that is rendered in the direction into the screen.  

The quality of the shading in the rendering is lowered if the clip planes are
chosen very far apart.  By default, the clip planes are set automatically so as
to encompass all the atoms.  

The setting is effective after {\bf Apply} or {\bf Done} is pressed.

\subsubsection{Slicing...}

Click on {\bf Slicing...} to open a small dialog window that allows the manual
setting of pairs of planes parallel to the coordinate axes.  If slicing is
activated by the toggle button labeled ''Slicing On'', only those atoms are
rendered that are located between pairs of slice planes. 

Slicing works independently for all three coordinate axis.   To adjust the
position of the planes, slicing can be switched off temporarily, and instead
the planes can be rendered along with the atoms.  This is done using the toggle
buttons ''Indicate Only''.    

At this stage, slicing along directions not parallel to the coordinate axes is
not supported.

The setting is effective after {\bf Apply} or {\bf Done} is pressed.
 
\subsection{Settings Menu}

\subsubsection{Hide Axis}

Click on {\bf Hide Axis} to inhibit the rendering of the yellow coordinate
system indicating directions. 

Click on {\bf Show Axis} to get back to the original state.  

\subsubsection{Hide Contour}

Click on {\bf Hide Contour} to remove the red contour outline that is 
drawn in the rendering.

Click on {\bf Show Contour} to get back to the original state.  

\subsubsection{Only Contour}

Click on {\bf Only Contour} to render only the red outline, and none 
of the particles.  This is useful to find suitable view parameters for a 
very large set of particles.

Click on {\bf Contour And Particles} to get back to the original state.  

\subsection{Data Menu}

\subsubsection{Translate...}

Click on {\bf Translate...} to open a small dialog window to translate 
the data set in space.  The translation vector is given in the text windows
labeled {\bf Translate X}, {\bf Translate Y}, and {\bf Translate Z}.  
Alternatively, translations can be specified as {\bf Translate Right}, 
{\bf Translate Top}, and {\bf Translate Forward}.  

The translation is effective after {\bf Apply} or {\bf Done} is pressed.

A button labeled {\bf Reset Button} is used to translate the data set 
back to the origin.

\subsubsection{Swap}

The entries {\bf Swap XY}, {\bf Swap XZ}, and {\bf Swap YZ} are used 
to internally swap the coordinate values with each other, so that for 
instance a particle at the position ($x_0$, $y_0$, $z_0$) is moved to 
the position ($y_0$, $x_0$, $z_0$). 

\subsubsection{Mirror}

The entries {\bf Mirror X}, {\bf Mirror Y}, and {\bf Mirror Z} are used 
to internall change the sign of the coordinate values, so that for 
instance a particle at the position ($x_0$, $y_0$, $z_0$) is moved to 
the position ($-x_0$, $y_0$, $z_0$). 

\subsubsection{Stretch...}

This entry pops up a window with three text windows, one for each 
coordinate.  In each window a factor can be specified with which 
all coordinates are multiplied during rendering.  A factor greater
then 1 means that the rendered structure is extended in the direction
at hand. 

Buttons next to the text windows can be used to reset the entries
to their default values of 1.

The setting is active after {\bf Apply} or {\bf Done} is pressed.

\subsection{Help Menu}

\subsubsection{About...}

The entry {\bf About...} produces a small window with copyright information.

\subsubsection{License}

The entry {\bf About...} produces a small window with licensing information.
The full GNU license text is printed in the window from which AViz was started.

\subsubsection{Distribution}

The entry {\bf About...} produces a small window with information on 
distributing the software.  The full GNU distribution conditions are 
printed in the window from which AViz was started.

\section{AViz Graphics Window}

The graphics window is surrounded by a frame with controls to set view 
parameters such as viewing distance and rotation angle.  

You can also set view angles completely without controls, by left-clicking
and dragging the mouse.  Right-clicking changes the viewing distance, and 
center-clicking lets you translate the particles.

The frame sports three spin wheels by which you can rotate the rendering about 
the three axis of the fixed screen frame.  You can also rotate in small 
steps by using tool buttons adjacent to the wheels. 

A fourth spin wheel, with adjacent buttons, is used to adjust the viewing
distance.

An extra button labeled {\bf Auto} can be used to increment viewing parameters
continuously.  In combination with one or several of the tool buttons and the
{\bf AutoSnap} button, the {\bf Auto} button provides a very simple way to
create an animation showing a given structure from a continuously changing view
point.

The remaining tool buttons in graphics frame are used to assume the default
view parameters ({\bf Home}), to designate the current view parameters as
defaults ({\bf Set Home}), and to use {\bf perspective rendering} or {\bf
orthogonal rendering}.  Perspective view provides a more realistic rendering.
In orthogonal view, particles farther apart from the viewer do not appear
smaller in the rendering, and parallel lines are rendered as parallel lines.   

Finally, there are two tool buttons to select either {\bf black background} 
or {\bf white background}.

\section{File Lists}

File lists (see section \ref{fileList}) give you the possibility to leave
through a set of coordinate files, and in this way visualize the dynamics
in your simulation.  

\subsection{File List Dialog}
\label{fileListDialog}

The file list dialog pops up when you press the {\bf File List...} entry in the
{\bf File} menu.  

The current file list is indicated in a text window.  You may view the files in
the list in steps of 1, 10, or continously, by using the {\bf SingleStep}, {\bf
FastStep}, and {\bf Cycle} buttons, respectively.  Additional convenience
buttons let you jump to the beginning or end of the sequence.  Radio buttons
can be used to set the direction of the cycling.  

When you like the cycle, you may press the {\bf Auto Snap} button and wait for
your sequence of PNG files to be created.

As you render subsequent frames, the rendering parameters may be slightly
different from frame to frame.  The reason is that the rendering parameters
depend on the dimensions of the data set.  You can switch off repeated 
udpateds of the view parameter settings by clicking the {\bf Keep ViewScale} 
check box (see also section \ref{viewParamAnimation}).

You may also render tracks that show the motion of single particles during
the sequence; see section \ref{trackDialog}.

\subsection{Track Dialog}
\label{trackDialog}
You open the track dialog by pressing the {\bf Tracks...} button in the 
file list dialog. 

A selector is used to select the particle type to which the settings apply.
For each particle type, you can choose whether or not to render motion
tracks.  You may either render, at any stage, the complete track within
your sequence of data files, or the track covered up to the current
stage, starting from the first file.  

The tracks can be rendered with or without periodic boundary conditions
applied.  The minimum and maximum extensions of the data set are taken to be
the boundaries if periodic boundary conditions are valid.

The setting is active after {\bf Apply} or {\bf Done} is pressed.

\section{Creating Animations}

AViz offers several possibilities to facilitate the 
creation of animations.  

Here is a very simple way to create an animation in which a three-dimensional
structure is visualized by rotating it slowly: Set up a good view point and
decent atom colors, then press {\bf Auto Snap} and one of the rotation tool
buttons in the graphics frame, and let AViz do its work.  When the rotation is
completed, you will end up with a sequence of PNG image files that can be
turned into an animation, using MPEG-encoding software. 

If you have a sequence of XYZ coordinate data files and want to turn 
them into an animation, using the same view point all the time, you 
may create a file list as follows:

\begin{verbatim}
ls -1 *.xyz |sort -n > filelist.dat
\end{verbatim}

A similar file list is created by using the convenience button labeled {\bf
Generate File List} in the file browser (see section \ref{fileOpen}).

Then, use the {\bf File/Open/Open XYZ File List} menu entry to read in the file
{\tt filelist.dat}.  Use the file list dialog to control the rendering of a
sequence of frames.  In combination with the {\bf Auto Snap} button, you then
create a sequence of PNG image files that can be turned into an animation. 

Alternatively, you may create a simple shell script that consists of repeated
entries of the form:
\begin{verbatim}
aviz -snapq aluOx1.xyz -geometry 902x305+0+0
aviz -snapq aluOx2.xyz -geometry 902x305+0+0
aviz -snapq aluOx3.xyz -geometry 902x305+0+0
aviz -snapq aluOx4.xyz -geometry 902x305+0+0
\end{verbatim}

When this script is running, it will start up AViz repeatedly.  AViz 
will render the specified coordinate file, take a PNG snaphot of the 
rendering, and kill itself.  (Note: Offscreen rendering is not yet 
implemented....)

\subsection{Watching a File}
\label{watching}
In some situations one would like to monitor an on-going simulation.  Here
the {\bf -watch} option may be of use.  When AViz is started up as in 

\begin{verbatim}
aviz -watch wedgeSim.xyz 
\end{verbatim}

it will render the data set in the specified file.  Every 10 seconds or so, 
AViz will re-examine the file.  If it has been modified in the mean time, 
AViz will automatically re-read the file and udpate the rendering.  If 
the {\bf Auto Snap} button is depressed, a sequence of PNG images is generated
as a byproduct.

\subsection{View Parameters in Animations}
\label{viewParamAnimation}
Note that the view parameters do not completely specify how a data set is 
rendered -- it also depends on the extension of the data set.  If you try 
to animate a sequence of sets with differing extensions, you will end up 
with an uneven and jumpy animation.  The solution is to add ''dummy'' atoms
at fixed positions, sufficiently remote from the main set.  These dummy 
atoms will then define a constant outline of the set.  They must not be 
rendered, of course.  

If your animation is based on a file list, you may also inhibit view parameter 
updates by using the {\bf Keep ViewScale} button in the file list dialog 
(see section \ref{fileListDialog}).

\subsection{Creating AnimatedGIF Files}
\label{creatingAnimatedGIF}
The menu entry {\bf Animation...} produces a convenience dialog window that 
lets you generate an AnimatedGIF file based on image files in PNG-format.  
The dialog window contains a text field in which the directory containing
image files must be given.  You can use the {\bf Browse} button to select
the directory.  

Once the directory is entered, you press {\bf Create AnimatedGIF}, and let AViz
do the rest.  

This is a convenience function of experimental character.  It relies on the
image processing program {\tt convert} which is quite common on Linux systems

\section{Frequently Asked Questions}

{\em Why does AViz produce PNG image files and not GIF files, which are much 
more common?}

There appears to be a legal problem with using the GIF image file format.  
For this reason, there is a tendency in the Linux community to move over 
to the PNG format.  It may be expected that the PNG format will rise 
quickly in popularity, as more and more software is supporting it.  In the 
mean time, there exists a lot of excellent image format conversion software, 
most notably the {\bf convert}, which is part of the freely-available 
ImageMagick package.

\end{document}
